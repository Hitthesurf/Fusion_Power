\documentclass[12pt,a4paper]{article}

\setlength{\topmargin}{0.0in}
\setlength{\oddsidemargin}{0.33in}
\setlength{\textheight}{9.0in}
\setlength{\textwidth}{6.0in}
\renewcommand{\baselinestretch}{1.25}

\usepackage[backend=bibtex,style=numeric,sorting=none]{biblatex}
\usepackage{amsmath}
\usepackage{amsfonts}

\addbibresource{ref.bib}

\title{Title of Special Topic}
\author{Lecture Course}
\date{Candidate Number: xxxxxx}

\begin{document}

\maketitle

\thispagestyle{empty}

\newpage
\setcounter{page}{1}



\begin{abstract}
  Insert some waffle here if you want. Abstracts are optional for
  special topics but are part of page count.
\end{abstract}

\section{Introduction}
In this paper, we aim to find robust methods to solve highly anisotropic diffusion problems. Thus we want to solve

\begin{equation}
\begin{cases}
-\nabla \cdot (\mathbb{A}\nabla u) = f, & \text{ in }\Omega,\\
\mathbf{n}\cdot \mathbb{A}\nabla u = 0, & \text{ on }\Gamma_N, \\
u = g, & \text{  on }\Gamma_D,
\end{cases}
\end{equation}
where
\begin{equation}
\mathbb{A} = \epsilon^{-1} A_{||}\mathbf{b}\otimes \mathbf{b}
+(I - \mathbf{b}\otimes \mathbf{b})\mathbb{A}_{\perp}
(I - \mathbf{b}\otimes \mathbf{b}).
\end{equation}
    Here $\epsilon \ll 1$ is the anisotropic coefficient. And $|\mathbf{b}|=1$ with $\mathbf{b}$ representing the magnetic field.
State restriction on the b

From multiple papers the best notation for this problem is stated in \cite{} \cite{}. Thus we will use a similar notation. 

Talk about problem (problem from not being well posed.)

Say $\Lambda_{in}$ can be any part of stream line


For a scaler feild $u\in\mathbb{R}$ and vector field $v \in \mathbb{R}^d$, we use the notation
\begin{align}
v_{||} &=(\mathbf{b} \otimes \mathbf{b})v = (v \cdot \mathbf{b})\mathbf{b}
, & v_{\perp} &= (\mathbb{I}-\mathbf{b} \otimes \mathbf{b})v,\\
\nabla_{||}u &= (\mathbf{b} \otimes \mathbf{b}) \nabla u = (\nabla u \cdot \mathbf{b})\mathbf{b},
& \nabla_{\perp} u &= (\mathbb{I}-\mathbf{b} \otimes \mathbf{b}) \nabla u,\\
\nabla_{||} \cdot u &= \nabla \cdot u_{||},
& \nabla_{\perp} \cdot u &= \nabla \cdot u_{\perp}.
\end{align}
Now we define 
\begin{align}
\Delta_{||}u &= \nabla_{||}\cdot(A_{||}\nabla_{||}u)  =
\nabla \cdot(A_{||} (\mathbf{b} \otimes \mathbf{b}) \nabla u)\\
\Delta_{\perp}u  &= \nabla_{\perp}\cdot(\mathbb{A}_{\perp}\nabla_{\perp}u)  = 
\nabla \cdot((\mathbb{I}-\mathbf{b} \otimes \mathbf{b})\mathbb{A}_{\perp}(\mathbb{I}-\mathbf{b} \otimes \mathbf{b})\nabla u)
\end{align}

Thus from \ref{} and \ref{} it is clear to see \ref{} can be written as
\begin{align}
\mathbb{A} \nabla u = 
\varepsilon^{-1}(A_{||}\nabla_{||} u)_{||} + 
(\mathbb{A}_{\perp}\nabla_{\perp}u)_{\perp} \\
 \\
\end{align}
Thus putting \ref{} into the PDE \ref{} we get
\begin{equation}
\begin{cases}
-\varepsilon^{-1} \Delta_{||}u - \Delta_{\perp}u = f, & \text{ in }\Omega,\\
\mathbf{n}\cdot \mathbb{A}\nabla u = 0, & \text{ on }\Gamma_N, \\
u = g, & \text{  on }\Gamma_D,
\end{cases}
\end{equation}

With the equation (\ref{}) being equal because
\begin{equation}
\nabla_{||}\cdot(A_{||}\nabla_{||}u) = 
\nabla \cdot (A_{||})_{||} 
\end{equation}
can be done as $A_{||}$ is scalar
and (\ref{}) being equal because 

We can now write the PDE \ref{} in weak form

\begin{align}
a_{||}(\alpha, \beta) &= \int_{\Omega} A_{||} \nabla_{||}\alpha \cdot \nabla_{||}\beta d\mathbf{x} \\
a_{\perp}(\alpha, \beta) &= \int_{\Omega}(\mathbb{A}_{\perp} \nabla_{\perp}\alpha )\cdot \nabla_{\perp} \beta d\mathbf{x}
\end{align}
This can be easily calculated in firedrake using
CODESNIP

By splitting up the PDE \ref{} into parts parrelle and perpendicular to the vector feild we get the PDE 

b repressents column vector 
Now we will define all the spaces we will be using

Need to explain why this is a problem

Now we assume $b$ is perp to lam_N and $b$ parralle to lam_D

It should be noted that that ___ are a lot easier to discretise. ()

\section{Overview Of Current Methods}
For these methods and for the numerical simulations in section \ref{} we will take the boundary conditions
\begin{equation}
\begin{cases}
\mathbf{n}\cdot \mathbb{A}\nabla u = 0, & \text{ on }\Gamma_N, \\
u = g, & \text{  on }\Gamma_D.
\end{cases}
\end{equation}

\subsection{Singular Pertubation Method} \label{SP}
The single pertubation method involves solving the weak form of equation \ref{}. Therefore we have a trial function $u \in $ and a test function $\psi \in$. Thus we solve the weak form (\ref{}) for $(u) \in $.
\begin{equation}
(SP)
\begin{cases}
\varepsilon^{-1}a_{||}(u, \psi) + a_{\perp}(u, \psi) = \int_{\Omega} f \psi d\mathbf{x}
\end{cases}
\end{equation}
This method excells for when $\varepsilon$ is normal size

\subsection{Limit Method} \label{LM}
We now introduce a limit method discussed in paper \cite{}. The limit method involves solving the the equation \ref{} when $\varepsilon \rightarrow 0$. This leads to solving the weak form \ref{} for the trial functions $(u, q) \in $ 
\begin{equation}
(LM)
\begin{cases}
a_{\perp}(u, \hat{u}) + a_{||}(q, \hat{u}) = \int_{\Omega} f \hat{u} d\mathbf{x}, 
&\forall \hat{u} \in \\
a_{||}(u, \hat{q}) = 0, & \forall \hat{q} \in 
\end{cases}
\end{equation}

where $(\hat{u}, \hat{q}) \in $ are test functions.

\subsection{MMAP Method} \label{MMAP}
We now introduce a Micro-Macro Asymptotic-Preserving method discussed in paper \cite{}. This method is similar to the $(LM)$ method from section \ref{LM} but from the additional $\varepsilon$ term it is able to deal with $\varepsilon \approx 1$. This leads to solving the weak form \ref{} for the trial functions $(u, q) \in $ 
\begin{equation}
(MMAP)
\begin{cases}
a_{\perp}(u, \hat{u}) + a_{||}(q, \hat{u}) = \int_{\Omega} f \hat{u} d\mathbf{x}, 
&\forall \hat{u} \in \\
a_{||}(u, \hat{q}) - \varepsilon a_{||}(q, \hat{q}) = 0, & \forall \hat{q} \in 
\end{cases}
\end{equation}

where $(\hat{u}, \hat{q}) \in $ are test functions.

\subsection{AP Method} \label{AP}
We now discuss an Asymptotic-Preserving method stated in paper \cite{}. This method involves solving the weak form (\ref{}) for trial functions $(p, \lambda, q, \ell, \mu) \in $

\begin{equation}
(AP)
\begin{cases}
a_{\perp}(p, \eta)+a_{\perp}(q, \eta) + a_{||}(\eta, \lambda) &= \int_{\Omega}f \eta d\mathbf{x},
\\
 a_{||}(p, \kappa) &= 0, 
 \\
 a_{||}(q, \xi) + \varepsilon a_{\perp}(p+q, \xi) &= 
 \int_{\Omega} (\varepsilon f -\ell)\xi d\mathbf{x},
 \\
 a_{||}(\chi, \mu) + \int_{\Omega}q \chi d\mathbf{x} &= 0,
 \\
 a_{||}(\ell, \tau) &= 0,
 \\
\end{cases}
\end{equation}
where $(\eta, \kappa, \xi, \chi, \tau)\in$ are test functions.


\subsection{Limit Stabilisation Method} \label{LM_STAB}
The $(LM)$ method stated in section \ref{LM} struggles when there is a magnetic field $\mathbf{b}$ with closed lines. Thus we introduce a stabilisation method which is discussed in \cite{}. We implement this modification to get the weak form
\begin{equation}
(LM\_STAB)
\begin{cases}
a_{\perp}(u, \hat{u}) + a_{||}(q, \hat{u}) = \int_{\Omega} f \hat{u} d\mathbf{x}, 
&\forall \hat{u} \in \\
a_{||}(u, \hat{q}) = \sigma \int_{\Omega} q \hat{q} d\mathbf{x}, & \forall \hat{q} \in 
\end{cases}
\end{equation}
We solve for trial functions $(u, q) \in $ with test functions $(\hat{u}, \hat{q}) \in $. Say what sigma is proportional too

\subsection{MMAP Stabilisation Method} \label{MMAP_STAB}
\begin{equation}
(MMAP\_STAB)
\begin{cases}
a_{\perp}(u, \hat{u}) + a_{||}(q, \hat{u}) = \int_{\Omega} f \hat{u} d\mathbf{x}, 
&\forall \hat{u} \in \\
a_{||}(u, \hat{q}) - \varepsilon a_{||}(q, \hat{q}) = \sigma \int_{\Omega} q \hat{q} d\mathbf{x}, & \forall \hat{q}\in 
\end{cases}
\end{equation}

\subsection{Deluzet-Narski Method} \label{DN}
This method turns the strongly anisotropic equation into two mildly anisotropic equations to be solved by iteration. This method is stated in \cite{}. This introduces $\varepsilon_0 \gg \varepsilon$ and gives the weak form \ref{} which solves for the sequence of trial functions $(u^{(n)}, q^{(n)}) \in $ .
 \begin{equation}
 (DN)
   \begin{cases}
  a_{||}(u, \hat{u}) + \varepsilon_0 a_{\perp}(u, \hat{u}) = 
  \varepsilon_0 \int_{\Omega} f \hat{u} d\mathbf{x}
  +(\varepsilon-\varepsilon_0)a_{||}(q, \hat{u}),
  & \forall \hat{u} \in \\
  a_{||}(q, \hat{q}) + \varepsilon_0 a_{\perp}(q, \hat{q}) = 
  \int_{\Omega} f \hat{q} d\mathbf{x}
  -a_{\perp}(u - \varepsilon_0 q, \hat{q}),
  & \forall \hat{q} \in 
  \end{cases}
  \end{equation}
With test functions $(\hat{u}, \hat{q}) \in \mathcal{V} \times \mathcal{V}$.

\section{New Method}
We introduce a new method based on unpublished notes from Patrick Farrell \cite{}. 
By doing the substitution $q=\varepsilon^{-1} \mathbf{b} \cdot \nabla u$ into (\ref{}) we get the weak form
\begin{equation}
(PF)
\begin{cases}
\int_{\Omega}A_{||}q\mathbf{b} \cdot \nabla \hat{u}  d\mathbf{x} + a_{\perp}(u, \hat{u}) 
= \int{\Omega} f \hat{u} d\mathbf{x},
\\
\int_{\Omega}\mathbf{b} \cdot \nabla \hat{u}\hat{q} d\mathbf{x} 
= \varepsilon \int_{\Omega}q\hat{q} d \mathbf{x}, 
\end{cases}
\end{equation}
We solve for trial function with test functions . Also, after some algebraic manipulation 
\begin{align}
q &= \varepsilon^{-1} \mathbf{b}\cdot \nabla u\\
q &= \varepsilon^{-1} \mathbf{b} \cdot \nabla_{||} u \\
\mathbf{b}q &= \varepsilon^{-1} \nabla_{||} u 
\end{align}
Additionally, we can use the stabilisation technique stated in \cite{} to get the weak form
\begin{equation}
(PF\_STAB)
\begin{cases}
\int_{\Omega}A_{||}q\mathbf{b} \cdot \nabla \hat{u}  d\mathbf{x} + a_{\perp}(u, \hat{u}) 
= \int{\Omega} f \hat{u} d\mathbf{x},
\\
\int_{\Omega}\mathbf{b} \cdot \nabla \hat{u}\hat{q} d\mathbf{x} 
= (\varepsilon + \sigma)\int_{\Omega}q\hat{q} d \mathbf{x}, 
\end{cases}
\end{equation}

\section{Derivation Of Current Methods}
It is important to understand the derivation of the current methods because the techniques used ...



\section{Numerical Demonstrations}
In these examples we calculate $f$ from passing $u$ through the PDE.
\subsection{Example 1}
We will solve the PDE (\ref{}) with $\Omega = [0,1]^2, \Gamma_D = \{y=0 \text{ or } y=1\}$ and $\Gamma_N = \{x=0 \text{ or } x=1\}$. We chose a magnetic field such
\begin{equation}
\mathbf{b} = \frac{\mathbf{B}}{|\mathbf{B}|}, 
\mathbf{B} = \left[ \begin{matrix}
\alpha(2y-1)cos(m\pi x) + \pi\\
\pi \alpha m (y^2-y)sin(m \pi x)
\end{matrix} \right]
\end{equation}
Thus we have $\Gamma_{in} := \{x=0\}$ and exact solution
\begin{equation}
u = \sin(\pi y + \alpha (y^2-y) \cos(m\pi x)) + \varepsilon \cos(2 \pi x) \sin(\pi y)
\end{equation}

nice pictures with vector field
calculation of clossed field line
have vector field of 3 examples (3)
have f and the exact solution at different eps (12)

\subsection{Example 1, Numerical Demostration}
Show error plots for example for ranging eps  L and H (6)
Show stabalisation gives same
Do DN plot (1)
Changing grid size table 
size | dof | L2 error | H1 error PF, MMAP
Show gamma in can be on any part of streamline

\section{Example 2}
We now propose an example on an annulus woth closed field lines.
We will solve the PDE (\ref{}) with $\Omega := \{x,y \in \mathbb{R}, 1<x^2+y^2<4\}$, $\Gamma_D := \{x,y \in \mathbb{R}, (x^2+y^2=1 \text{ or } x^2+y^2=4)\}$ and $\Gamma_N := \emptyset$. We chose the magnetic field 
\begin{equation}
\mathbf{b} = \frac{\mathbf{B}}{|\mathbf{B}|} = 
\left[ \begin{matrix}
-y\\
 x
\end{matrix} \right]/\sqrt{x^2+y^2}, 
\mathbf{B} = \left[ \begin{matrix}
-y\\
 x
\end{matrix} \right]
\end{equation}
We will use solution
\begin{equation}
u = (1+\varepsilon)(x^2 + y^2 -1)(4-x^2-y^2)
\end{equation}
From Fig \ref{} it can be seen the magnetic field have closed lines. Therefore, for methods which involve $\Gamma_{in}$ we use the stabilisation variants. However, from the numerical investigation of Example 1 we found that we can change $\Gamma_{in}$ to $\Gamma_{out}$ and get similar error. This strongly suggests that $\Gamma_{in}$ can represent a point on each streamline. Thus we will numerically investigate the consequence of setting $\Gamma_{in}:=\{x,y \in \mathbb{R}, x>0, y=0\}$.




Therefore, these results strongly suggests that we can apply the condition  will investigate the numerical solution when $\Gamma_{in}$ is applied to part of the streamline, as in the numerical

\section{Example 3, Magnetic Islands}
We now solve an example on a square domain with non-zero boundary conditions on $\Gamma_D$ and with closed and open field lines.
We will solve the PDE (\ref{}) with $\Omega = [0,1]^2, \Gamma_D = \{y=0 \text{ or } y=1\}$ and $\Gamma_N = \{x=0 \text{ or } x=1\}$. We chose a magnetic field such 
\begin{equation}
\mathbf{b} = \frac{\mathbf{B}}{|\mathbf{B}|}, 
\mathbf{B} = \left[ \begin{matrix}
-\cos(\pi y)\\
4a \sin(4 \pi x)
\end{matrix} \right]
\end{equation}
We will use exact solution
\begin{equation}
u = \sin(10\sin(\pi y)-10a\cos(4 \pi x)) + \varepsilon \cos(2 \pi x)\sin(10 \pi y)
\end{equation}

\section{Example 4, Magnetic Islands}
Why do an example with zero BC ?
We now introduce an example on a unit square where $\Gamma_D$ has zero boundary conditions. Additionally, this has open and closed field lines.
Thus on $\Omega = [0,1]^2, \Gamma_D = \{y=0 \text{ or } y=1\}$ and $\Gamma_N = \{x=0 \text{ or } x=1\}$ we solve the PDE (\ref{}) with the same magnetic
\begin{equation}
\mathbf{b} = \frac{\mathbf{B}}{|\mathbf{B}|}, 
\mathbf{B} = \left[ \begin{matrix}
-\cos(\pi y)\\
4a \sin(4 \pi x) \sin(\pi y)
\end{matrix} \right]
\end{equation}
Where the exact solution is 
\begin{equation}
u = \sin(10 \sin(\pi y) \exp^{-a\cos(4 \pi x)}) 
\end{equation}

\section{Example 5, Torus}
We now look at an example in 3D where the domain $\Omega $ is a torus with inner radius $r_I = 1$, outer radius $r_O = 0.5$ and is centred at the origin. It will have zero Dirichlet boundary conditions. For this eaxmple we will use Toroidal coordinates which are explained in Appendix \ref{} and demonstrate how to find the inverse map. We invert back to cartesian cordinates because the toroidal laplacian \cite{} is very complicated. Thus the definition of $r$ is
\begin{equation}
r = \sqrt{r_I^2 + x^2 + y^2 + z^2 -2r_I\sqrt{x^2+y^2}}
\end{equation}
Where $r$ denotes the minimum distance from the circulfurance of a circle centered at the origin with radius $r_I$ and has $z=0$. We have magnetic field 
\begin{equation}
\mathbf{b} = \frac{\mathbf{B}}{|\mathbf{B}|} = 
\left[ \begin{matrix}
-y\\
 x \\
 0
\end{matrix} \right]/\sqrt{x^2+y^2}, 
\mathbf{B} = \left[ \begin{matrix}
-y\\
 x\\
 0
\end{matrix} \right]
\end{equation}
Also, we will have exact solution
\begin{equation}
u = 
\begin{cases}
(1+\varepsilon)((0.5)^2 - r ),\\
(1+\varepsilon)(0.25 -  \sqrt{r_I^2 + x^2 + y^2 + z^2 -2r_I\sqrt{x^2+y^2}})
\end{cases}
\end{equation}

\section{Methods For Investigation}

\subsection{Line Integration}


\subsection{Mapping of f}


\subsection{Splitting of domain}

\printbibliography[heading=bibintoc]

\appendix

\section{Title of Appendix}

Appendices are definitely not necessary and assessors are not obliged to read them so only use them for non-vital text, figures or calculations.$\phi$

\subsection{Toroidal Coordinates}
Here we discuss how to parameterise a torus and to find its inverted map. We use the Figures \ref{} \ref{} below to help us derive the paramterisation.


From visual inspection of Figure \ref{} we get 
\begin{align}
x &= (r_I + r_{xy})\cos(\theta)\\
y &= (r_I + r_{xy})\sin(\theta)
\end{align}
Additionally, from Figure \ref{} we get 
\begin{align}
r_{xy}& = r \cos(\phi) \\
z &= r \sin(\phi)
\end{align}
Thus joining the equations (\ref{}) to (\ref{}) we get the toroidal parametrisation
\begin{align}
x &= (r_I + r\cos(\phi))\cos(\theta) \\
y &= (r_I + r \cos(\phi))\sin(\theta) \\
z &= r \sin(\phi)
\end{align}
where $0\leq r \leq r_O$ and $0 \leq \phi, \theta \leq 2 \pi$. Now we calculate the inverse of this map by considering $x^2 + y^2 + z^2$. 
\begin{align}
x^2 + y^2 + z^2 &= (r_I + r \cos(\phi))^2 + r^2\sin^2(\phi)\\
&= r_I^2 + 2r_Ir\cos(\phi) + r^2\\
&= r_I^2 +2r_I \sqrt{r^2-z^2} + r^2
\end{align}
This is a quadratic in disguise thus after some algebraic manipulation we get $4$ possible solutions
\begin{equation}
r = \pm\sqrt{r_I^2 + x^2 + y^2 + z^2 \pm 2r_I\sqrt{x^2+y^2}}
\end{equation}
However, by using enforcing $r\geq0$ we remove two solutions. We find the final solution by substitution. We use the substitution $(x, y, z) = (r_I, 0, 0)$ where $r=0$. With $+$ we get $r=2r_I$ and for $-$ we get $r = 0$. Thus we have 
\begin{equation}
r = \sqrt{r_I^2 + x^2 + y^2 + z^2 - 2r_I\sqrt{x^2+y^2}}
\end{equation}
For completeness will we calculate $\theta$ and $\phi$. To get $\theta$ and $\phi$ we do a similar process used for calculating the argument of complex numbers. We note $r_{xy}=x^2+y^2-r_I$ thus we get
\begin{align}
\theta &= \text{atan2}(y, x) \\
\phi &= \text{atan2}(z, x^2+y^2-r_{I})
\end{align}
where atan2 is a common variation of the arctan function.

We use the inverse of the map because calculating the differential operators leads to large expressions. These expressions can be found in \cite{}. When we need to solve a PDE on a domain which can be parameterised it is usually easier to turn the source term $f$ into Cartesian coordinates so we deal with the Cartesian differential operators.

\subsection{Formulation of examples}
How i found the exact solution for problems 
\section{CG Order 2}
\subsection{Solve Linear System in Parrelle}


\end{document}