\documentclass[12pt]{ociamthesis}

\setlength{\topmargin}{0.0in}
\setlength{\oddsidemargin}{0.33in}
\setlength{\textheight}{9.0in}
\setlength{\textwidth}{6.0in}
\renewcommand{\baselinestretch}{1.25}

\usepackage[backend=bibtex,style=numeric,sorting=none]{biblatex}
\usepackage{amsmath}
\usepackage{amsfonts}
\usepackage{graphicx}
\usepackage{listings}
\usepackage{xcolor}
\usepackage{float}
\usepackage{subcaption}

\addbibresource{ref.bib}

%New colors defined below
\definecolor{codegreen}{rgb}{0,0.6,0}
\definecolor{codegray}{rgb}{0.5,0.5,0.5}
\definecolor{codepurple}{rgb}{0.58,0,0.82}
\definecolor{backcolour}{rgb}{0.95,0.95,0.92}

%Code listing style named "mystyle"
\lstdefinestyle{mystyle}{
  backgroundcolor=\color{backcolour}, commentstyle=\color{codegreen},
  keywordstyle=\color{magenta},
  numberstyle=\tiny\color{codegray},
  stringstyle=\color{codepurple},
  basicstyle=\ttfamily\footnotesize,
  breakatwhitespace=false,         
  breaklines=true,                 
  captionpos=b,                    
  keepspaces=true,                 
  numbers=left,                    
  numbersep=5pt,                  
  showspaces=false,                
  showstringspaces=false,
  showtabs=false,                  
  tabsize=2
}

%"mystyle" code listing set
\lstset{style=mystyle}


\title{Parameter-Robust Discretisations of Anisotropic 
Diffusion Problems}
\author{Mark Pearson}
\college{Linacre College}
%\renewcommand{\submittedtext}{Dissertation submitted in partial fulfilment of the requirements for the degree of}
\degree{M.Sc.\ in Mathematical Modelling and Scientific Computing}
\degreedate{Trinity Term 2020}

\begin{document}

\maketitle

\thispagestyle{empty}

\newpage

\begin{acknowledgements} 

Firstly, I would like to thank my supervisor Patrick Farrell. He provided me with guidance and direction for this project. Additionally, he contributed his unpublished notes on the problem.

Secondly, thanks is given to Dr Kathryn Gillow for providing a well organised course and constant support.

Finally, I thank Culham Centre for Fusion Energy for funding this project. 

\end{acknowledgements}

\newpage

\begin{abstract}
In this paper, we wish to solve highly anisotropic diffusion problems on a domain $\Omega$. We discuss existing methods, propose new methods and demonstrate how to implement these methods in Python using Firedrake. Then we test and compare the different methods on a range of examples.
\end{abstract}

\newpage
\setcounter{page}{1}

\tableofcontents


\chapter{Introduction and Current Methods}
\section{Introduction}
In this paper, we aim to find robust methods to solve highly anisotropic diffusion problems. Thus we want to solve

\begin{equation} \label{PDE}
\begin{cases}
-\nabla \cdot (\mathbb{A}\nabla u) = f, & \text{ in }\Omega,\\
\mathbf{n}\cdot \mathbb{A}\nabla u = 0, & \text{ on }\Gamma_N, \\
u = 0, & \text{  on }\Gamma_D,
\end{cases}
\end{equation}
where $\Gamma_N \cap \Gamma_D := \partial \Omega$ and
\begin{equation} \label{Mat_A}
\mathbb{A} = \epsilon^{-1} A_{||}\mathbf{b}\otimes \mathbf{b}
+(I - \mathbf{b}\otimes \mathbf{b})\mathbb{A}_{\perp}
(I - \mathbf{b}\otimes \mathbf{b}).
\end{equation}
    Here $\epsilon \ll 1$ is the anisotropic coefficient. And $\mathbf{b}$ represents the electromagnetic vector field. From \cite{DN} we have restrictions on $\mathbf{b}$ they demand that
    \begin{enumerate}
  \item $\mathbf{b} \in \mathcal{C}^{\infty}(\Omega; \mathrm{R}^d);$.
  \item $|\mathbf{b}|=1$ for all $\mathbf{x} \in \Omega$;
  \item $\mathbf{b}\cdot \mathbf{n} = 0$ on $\Gamma_D$ the Dirichlet boundary;
  \item $\mathbf{b}\cdot \mathbf{n} \neq 0$ on $\Gamma_N$ the Neumann boundary;
\end{enumerate}
where $\mathbf{n}$ is the outward normal.

The main difficulty with solving this problem is that it is singular. Therefore, it is ill-conditioned for small $\varepsilon$ and when discretised we get a linear system that has an unbounded condition number. Thus we aim to find discretisations that are well posed.  

From multiple papers the best notation for this problem is stated in \cite{DN} and \cite{AP}. Thus we will use a similar notation. For a scalar field $u\in\mathbb{R}$ and vector field $v \in \mathbb{R}^d$, we use the notation
\begin{align}
v_{||} &=(\mathbf{b} \otimes \mathbf{b})v = (v \cdot \mathbf{b})\mathbf{b}
, & v_{\perp} &= (\mathbb{I}-\mathbf{b} \otimes \mathbf{b})v,\\
\nabla_{||}u &= (\mathbf{b} \otimes \mathbf{b}) \nabla u = (\nabla u \cdot \mathbf{b})\mathbf{b},
& \nabla_{\perp} u &= (\mathbb{I}-\mathbf{b} \otimes \mathbf{b}) \nabla u,\\
\nabla_{||} \cdot v &= \nabla \cdot v_{||},
& \nabla_{\perp} \cdot v &= \nabla \cdot v_{\perp}.
\end{align}
Now we define 
\begin{align} \label{Lap_para}
\Delta_{||}u &= \nabla_{||}\cdot(A_{||}\nabla_{||}u)  =
\nabla \cdot(A_{||} (\mathbf{b} \otimes \mathbf{b}) \nabla u)\\ \label{Lap_perp}
\Delta_{\perp}u  &= \nabla_{\perp}\cdot(\mathbb{A}_{\perp}\nabla_{\perp}u)  = 
\nabla \cdot((\mathbb{I}-\mathbf{b} \otimes \mathbf{b})\mathbb{A}_{\perp}(\mathbb{I}-\mathbf{b} \otimes \mathbf{b})\nabla u)
\end{align}

Thus from (\ref{Lap_para}) and (\ref{Lap_perp}) it is clear to see (\ref{Mat_A}) can be written as
\begin{align} \label{A_grad_u}
\mathbb{A} \nabla u = 
\varepsilon^{-1}(A_{||}\nabla_{||} u)_{||} + 
(\mathbb{A}_{\perp}\nabla_{\perp}u)_{\perp} 
\end{align}
Thus putting (\ref{A_grad_u}) into the PDE (\ref{PDE}) we get
\begin{equation}
\begin{cases}
-\varepsilon^{-1} \Delta_{||}u - \Delta_{\perp}u = f, & \text{ in }\Omega,\\
\mathbf{n}\cdot \mathbb{A}\nabla u = 0, & \text{ on }\Gamma_N, \\
u = 0, & \text{  on }\Gamma_D,
\end{cases}
\end{equation}
We now write the PDE (\ref{PDE}) in weak form
\begin{align}
\int_{\Omega}f\hat{u}d\mathbf{x} &= - \int_{\Omega} \nabla \cdot (\mathbb{A} \nabla u) \hat{u} \mathbf{x}\\
&= - \int_{\partial \Omega} \hat{u} (\mathbb{A} \nabla u) \cdot \mathbf{n} d\mathbf{x}
+ \int_{\Omega}(\mathbb{A}\nabla u)\cdot \nabla \hat{u} d \mathbf{x}\\
&= - \int_{\Gamma_D} \hat{u} (\mathbb{A}\nabla u) \cdot \mathbf{n} + 
\varepsilon^{-1}\int_{\Omega} A_{||} \nabla_{||}u \cdot \nabla_{||}\hat{u} d\mathbf{x} +
\int_{\Omega}(\mathbb{A}_{\perp} \nabla_{\perp} u )\cdot \nabla_{\perp} \hat{u} d\mathbf{x}
\end{align}
Where $u$ is a trial function and $\hat{u}$ is a test function. To simplify notation we use
\begin{align}
a_{||}(\alpha, \beta) &= \int_{\Omega} A_{||} \nabla_{||}\alpha \cdot \nabla_{||}\beta d\mathbf{x} \\
a_{\perp}(\alpha, \beta) &= \int_{\Omega}(\mathbb{A}_{\perp} \nabla_{\perp}\alpha )\cdot \nabla_{\perp} \beta d\mathbf{x}
\end{align}
For are numerical simulations we use Firedrake \cite{Dragon} which written in code is 
\lstinputlisting[language=Python]{CodeSnips/a_para.py}
Also, we have 
\begin{equation}
\varepsilon^{-1} (A_{||}\nabla_{||}u)_{||}\cdot \mathbf{n} = \varepsilon^{-1} (A_{||}\nabla_{||}u)\cdot \mathbf{n} = \varepsilon^{-1} A_{||}(\nabla_u \cdot \mathbf{b}) \mathbf{b} \cdot \mathbf{n} = 0,
\end{equation}
on $\Gamma_D$ as restriction $3$ of the electromagnetic field $\mathbf{b}$ states $\mathbf{b}\cdot \mathbf{n}=0$. This means from (\ref{}) and using notation (\ref{}) and (\ref{}) we get (\ref{}) becomes
\begin{equation} \label{PDE_w}
\int_{\Omega}f\hat{u}d\mathbf{x} = -\int_{\Gamma_D}\hat{u}(\mathbb{A}_{\perp}\nabla_{\perp} u)_{\perp}\cdot \mathbf{n} d \mathbf{x} + 
\varepsilon^{-1} a_{||}(u,\hat{u})+a_{\perp}(u, \hat{u})\\
\end{equation}
In the integral over $\Gamma_D$ in (\ref{PDE_w}) $u$ can be replaced with $0$ because in the boundary conditions for (\ref{PDE}) $u,\hat{u} \in \mathcal{V}$. This is included in the formula to show how one would calculate the problem with non zero boundary conditions.
Our methods require the use of the sets 
\begin{align}
\mathcal{H}^1 &:= \{\} \\
\Gamma_{in} &:= \{\mathbf{b} \in \Gamma : \mathbf{b} \cdot \mathbf{n} < 0\} \\
\mathcal{V} &:= \{v \in \mathcal{H}^1(\Omega) : v|_{\Gamma_{D}} = 0\} \\
\mathcal{L} &:= \{\lambda \in \mathcal{H}^1(\Omega) : \lambda |_{\Gamma_{in}\cup \Gamma_{D}} = 0\}
\end{align}

It should be noted that sets $\mathcal{V}$ and $\mathcal{L}$ are easy to discretise. However, from numerical simulations we find 

\section{Overview Of Current Methods}
 We now state some methods for solving the PDE (\ref{PDE}).
\subsection{Singular Pertubation Method} \label{SP}
The single perturbation method involves solving the weak form of equation (\ref{PDE}). Therefore we have a trial function $u \in \mathcal{V}$ and a test function $\psi \in \mathcal{V}$. Thus we solve the weak form (\ref{SP_w}) for $(u) \in \mathcal{V}$.
\begin{equation} \label{SP_w}
(SP)
\begin{cases}
\varepsilon^{-1}a_{||}(u, \psi) + a_{\perp}(u, \psi) = \int_{\Omega} f \psi d\mathbf{x}, \forall \psi \in \mathcal{V}
\end{cases}
\end{equation}
This method excells for when $\varepsilon$ is normal size

\subsection{Limit Method} \label{LM}
We now introduce a limit method discussed in paper \cite{AP}. The limit method involves solving the the equation (\ref{PDE}) when $\varepsilon \rightarrow 0$. Therefore, $\Delta_{||}u=0$ so since we have zero dirichlet boundary conditions the solution $u$ satisfies $\nabla_{||}u=0$ thus it is constant along the stream lines of the vector field. We apply this restriction using a Lagrangian parameter. This leads to solving the weak form (\ref{LM_w}) for the trial functions $(u, q) \in \mathcal{V} \times \mathcal{L}$ 
\begin{equation} \label{LM_w}
(LM)
\begin{cases}
a_{\perp}(u, \hat{u}) + a_{||}(q, \hat{u}) = \int_{\Omega} f \hat{u} d\mathbf{x}, 
&\forall \hat{u} \in \mathcal{V},\\
a_{||}(u, \hat{q}) = 0, & \forall \hat{q} \in \mathcal{L},
\end{cases}
\end{equation}
where $(\hat{u}, \hat{q}) \in \mathcal{V} \times \mathcal{L}$ are test functions.

\subsection{MMAP Method} \label{MMAP}
We now introduce a Micro-Macro Asymptotic-Preserving method discussed in paper \cite{MMAP}. This method is similar to the $(LM)$ method from section \ref{LM} but from the additional $\varepsilon$ term it is able to deal with $\varepsilon \approx 1$. This leads to solving the weak form (\ref{MMAP_w}) for the trial functions $(u, q) \in  \mathcal{V} \times \mathcal{L}$. 
\begin{equation} \label{MMAP_w}
(MMAP)
\begin{cases}
a_{\perp}(u, \hat{u}) + a_{||}(q, \hat{u}) = \int_{\Omega} f \hat{u} d\mathbf{x}, 
&\forall \hat{u} \in \mathcal{V}, \\
a_{||}(u, \hat{q}) - \varepsilon a_{||}(q, \hat{q}) = 0, 
&\forall \hat{q} \in \mathcal{L},
\end{cases}
\end{equation}
where $(\hat{u}, \hat{q}) \in \mathcal{V}\times \mathcal{L} $ are test functions.

\subsection{AP Method} \label{AP}
We now discuss an Asymptotic-Preserving method stated in paper \cite{AP}. This method involves solving the weak form (\ref{AP_w}) for trial functions $(p, \lambda, q, \ell, \mu) \in \mathcal{V} \times \mathcal{L}  \times \mathcal{V}  \times \mathcal{V}  \times \mathcal{L}$.

\begin{equation} \label{AP_w}
(AP)
\begin{cases}
a_{\perp}(p, \eta)+a_{\perp}(q, \eta) + a_{||}(\eta, \lambda) = \int_{\Omega}f \eta d\mathbf{x},
&\forall \eta \in \mathcal{V}\\
a_{||}(p, \kappa) = 0, 
&\forall \kappa \in \mathcal{L}\\
a_{||}(q, \xi) + \varepsilon a_{\perp}(p+q, \xi) = 
\int_{\Omega} (\varepsilon f -\ell)\xi d\mathbf{x},
&\forall \xi \in \mathcal{V}\\
a_{||}(\chi, \mu) + \int_{\Omega}q \chi d\mathbf{x} = 0,
&\forall \chi \in \mathcal{V}\\
a_{||}(\ell, \tau) = 0,
&\forall \tau \in \mathcal{L}\\
\end{cases}
\end{equation}
where $(\eta, \kappa, \xi, \chi, \tau)\in \mathcal{V} \times \mathcal{L}  \times \mathcal{V}  \times \mathcal{V}  \times \mathcal{L}$ are test functions. We have $u=p+q$.


\subsection{Limit Stabilisation Method} \label{LM_STAB}
The $(LM)$ method stated in section \ref{LM} struggles when there is a magnetic field $\mathbf{b}$ with closed lines. Thus we introduce a stabilisation method which is discussed in \cite{STAB}. We implement this modification to get the weak form
\begin{equation} \label{LM_STAB_w}
(LM\_STAB)
\begin{cases}
a_{\perp}(u, \hat{u}) + a_{||}(q, \hat{u}) = \int_{\Omega} f \hat{u} d\mathbf{x}, 
&\forall \hat{u} \in \mathcal{V}\\
a_{||}(u, \hat{q}) = \sigma \int_{\Omega} q \hat{q} d\mathbf{x}, & \forall \hat{q} \in \mathcal{V}
\end{cases}
\end{equation}
We solve for trial functions $(u, q) \in \mathcal{V} \times \mathcal{V}$ with test functions $(\hat{u}, \hat{q}) \in \mathcal{V} \times \mathcal{V}$. Say what sigma is proportional too
Additionally, we do not use $\mathcal{L}$ therefore there are no complications when the electromagnetic field $\mathbf{b}$ has closed field lines or field lines that converge to a point like $\dot{x} = x/|x|$ with $x=0$.

\subsection{MMAP Stabilisation Method} \label{MMAP_STAB}
We also apply the stabilisation method mentioned in \cite{STAB} to the $(MMAP)$ method stated in section \ref{MMAP}. Thus we get weak form
\begin{equation} \label{MMAP_STAB_w}
(MMAP\_STAB)
\begin{cases}
a_{\perp}(u, \hat{u}) + a_{||}(q, \hat{u}) = \int_{\Omega} f \hat{u} d\mathbf{x}, 
&\forall \hat{u} \in \mathcal{V}\\
a_{||}(u, \hat{q}) - \varepsilon a_{||}(q, \hat{q}) = \sigma \int_{\Omega} q \hat{q} d\mathbf{x}, & \forall \hat{q}\in \mathcal{V}
\end{cases}
\end{equation}
where we solve for trial functions $(u,q) \in \mathcal{V} \times \mathcal{V}$ with test functions $(\hat{u}, \hat{q}) \in \mathcal{V} \times \mathcal{V}$. 

\subsection{Deluzet-Narski Method} \label{DN}
This method turns the strongly anisotropic equation into two mildly anisotropic equations to be solved by iteration. This method is stated in \cite{DN}. This introduces $\varepsilon_0 \gg \varepsilon$ and gives the weak form \ref{DN_w} which solves for the sequence of trial functions $(u^{(n)}, q^{(n)}) \in \mathcal{V} \times \mathcal{V}$ .
 \begin{equation} \label{DN_w}
 (DN)
   \begin{cases}
  a_{||}(u^{(n+1)}, \hat{u}) + \varepsilon_0 a_{\perp}(u^{(n+1)}, \hat{u}) = 
  \varepsilon_0 \int_{\Omega} f \hat{u} d\mathbf{x}
  +(\varepsilon-\varepsilon_0)a_{||}(q^{(n)}, \hat{u}),
  & \forall \hat{u} \in \mathcal{V}\\
  a_{||}(q^{(n+1)}, \hat{q}) + \varepsilon_0 a_{\perp}(q^{(n+1)}, \hat{q}) = 
  \int_{\Omega} f \hat{q} d\mathbf{x}
  -a_{\perp}(u^{(n+1)} - \varepsilon_0 q^{(n)}, \hat{q}),
  & \forall \hat{q} \in \mathcal{V}
  \end{cases}
  \end{equation}
With test functions $(\hat{u}, \hat{q}) \in \mathcal{V} \times \mathcal{V}$.

\section{Proposed Methods}
We introduce a new method based on unpublished notes from Patrick Farrell \cite{}. This involves the substitution $q=\varepsilon^{-1} \mathbf{b} \cdot \nabla u$, after some algebraic manipulation we get the following are equivalent
\begin{align}
q &= \varepsilon^{-1} \mathbf{b}\cdot \nabla u,\\
q &= \varepsilon^{-1} \mathbf{b} \cdot \nabla_{||} u, \\
\mathbf{b}q &= \varepsilon^{-1} \nabla_{||} u.
\end{align}
Also, we introduce new sets 
\begin{align}
\mathcal{Q}_{in} &:= \{ q \in \mathcal{H}^1(\Omega) : q|_{\Gamma_{in}}=0\}, \\
\mathcal{Q} &:= \{ q \in \mathcal{H}^1(\Omega)\}.
\end{align}

\subsection{PF Method}
Therefore, by doing the substitution $q=\varepsilon^{-1} \mathbf{b} \cdot \nabla u$ into (\ref{PDE}) we get the weak form
\begin{equation} \label{PF_w}
(PF)
\begin{cases}
\int_{\Omega}A_{||}q\mathbf{b} \cdot \nabla \hat{u}  d\mathbf{x} + a_{\perp}(u, \hat{u}) 
= \int_{\Omega} f \hat{u} d\mathbf{x},
&\forall \hat{u} \in \mathcal{V}\\
\int_{\Omega}\mathbf{b} \cdot \nabla \hat{u}\hat{q} d\mathbf{x} 
= \varepsilon \int_{\Omega}q\hat{q} d \mathbf{x}, 
&\forall \hat{q} \in \mathcal{Q}_{in}.
\end{cases}
\end{equation}
We solve for trial functions $(u,q) \in \mathcal{V} \times \mathcal{Q}_{in}$ with test functions $(\hat{u}, \hat{q}) \in \mathcal{V} \times \mathcal{Q}_{in}$. 


\subsection{PF Stabilisation Method}
Additionally, we can use the stabilisation technique stated in \cite{STAB} to get the weak form
\begin{equation} \label{PF_STAB_w}
(PF\_STAB)
\begin{cases}
\int_{\Omega}A_{||}q\mathbf{b} \cdot \nabla \hat{u}  d\mathbf{x} + a_{\perp}(u, \hat{u}) 
= \int{\Omega} f \hat{u} d\mathbf{x},
&\forall \hat{u} \in \mathcal{Q}_D,\\
\int_{\Omega}\mathbf{b} \cdot \nabla \hat{u}\hat{q} d\mathbf{x} 
= (\varepsilon + \sigma)\int_{\Omega}q\hat{q} d \mathbf{x}, 
&\forall \hat{q} \in \mathcal{Q}.
\end{cases}
\end{equation}
We solve for trial functions $(u,q) \in \mathcal{V} \times \mathcal{Q}$ with test functions $(\hat{u}, \hat{q}) \in \mathcal{V} \times \mathcal{Q}$. 

\chapter{Numerical Demonstrations}
In these examples we calculate $f$ from passing $u$ through the PDE. Additionally, in these example we take $A_{||}=0$ and $\mathbb{A}_{\perp} = \mathbb{I}$. Where $\mathbb{I}$ is the identity matrix.
Explain how to calculte system of PDEs

Say use firedrake then do speed up test on GPU?

\section{Derivation of Lagrange Finite Element of Order $2$}
In our numerical simulations we use a continuous Lagrange order $2$ finite element. We now provide a method for the derivation of this finite element on an interval and a triangle. Additionally, following a similar procedure we can derive the basis functions for other domains. The resource \cite{defelement} states basis functions for many finite elements. This is not implemented in code we instead use the Firedrake implementation of CG order $2$. 

For each element we have basis functions $\phi_i(\mathbf{x})$ which have a linear combination from the set $\{1, x, x^2, y, y^2, z, z^2, xy, yz, xz\}$ and locations on domain $\ell_i$.  To find the coefficients of the linear combination we must solve
\begin{equation} \label{CG_eq}
\phi_i(\ell_j) =
\begin{cases}
1, &i=j,\\
0, &i\neq j.
\end{cases}
\end{equation}
It should be noted we can do a substitution to make the shapes unit size.
\subsection{Interval (1D)}
Therefore, we have $i \in \{0, 1, 2\}$ and basis functions 
\begin{equation}
\phi_i(x) = \xi_{i0} + \xi_{i1}x + \xi_{i2}x^2
\end{equation}
with $\ell_0 = 0, \ell_1 = 1$ and $\ell_2 = 0.5$. Thus by using (\ref{CG_eq}) we get equations to solve
\begin{align}
\phi_i(\ell_0) = \phi_i(0) &= \xi_{i0} &= \delta_{i0},\\
\phi_i(\ell_1) = \phi_i(1) &= \xi_{i0} + \xi_{i1} + \xi_{i2} &= \delta_{i1},\\
\phi_i(\ell_2) = \phi_i(0.5) &= \xi_{i0} + \frac{\xi_{i1}}{2} + \frac{\xi_{i2}}{4} &= \delta_{i2}
\end{align}
where $\delta$ represents Kronecker delta. This can be put into matrix form
\begin{equation}
\left[ \begin{matrix}
\xi_{i0} & \xi_{i1} & \xi_{i2}
\end{matrix} \right]
\left[ \begin{matrix}
1 & 1 & 1 \\
0 & 1 & 1/2 \\
0 & 1 & 1/4
\end{matrix} \right] = 
\left[ \begin{matrix}
\delta_{i0} & \delta_{i1} & \delta_{i2}
\end{matrix} \right]
\end{equation}
It is trivial to implement this into a matrix for all basis functions in this element.
\begin{equation}
\left[ \begin{matrix}
\xi_{00} & \xi_{01} & \xi_{02} \\
\xi_{10} & \xi_{11} & \xi_{12} \\
\xi_{20} & \xi_{21} & \xi_{22} 
\end{matrix} \right]
\left[ \begin{matrix}
1 & 1 & 1 \\
0 & 1 & 1/2 \\
0 & 1 & 1/4
\end{matrix} \right] = 
\left[ \begin{matrix}
1 & 0 & 0 \\
0 & 1 & 0 \\
0 & 0 & 1
\end{matrix} \right]
\end{equation}
Thus we get 
\begin{align}
\left[ \begin{matrix}
\xi_{00} & \xi_{01} & \xi_{02} \\
\xi_{10} & \xi_{11} & \xi_{12} \\
\xi_{20} & \xi_{21} & \xi_{22} 
\end{matrix} \right] &= 
\left[ \begin{matrix}
1 & 1 & 1 \\
0 & 1 & 1/2 \\
0 & 1 & 1/4
\end{matrix} \right]^{-1}, \\
\left[ \begin{matrix}
\xi_{00} & \xi_{01} & \xi_{02} \\
\xi_{10} & \xi_{11} & \xi_{12} \\
\xi_{20} & \xi_{21} & \xi_{22} 
\end{matrix} \right] &= 
\left[ \begin{matrix}
1 & -3 & 2 \\
0 & -1 & 2 \\
0 & 4 & -4
\end{matrix} \right],
\end{align}
This leads to the basis functions
\begin{align}
\phi_0(x) &= 2x^2-3x+1,\\
\phi_1(x) &= x(2x-1),\\
\phi_2(x) &= 4x(1-x)
\end{align}
With their visual representation in Figure \ref{InterFuncs}.
\begin{figure}[H]
     \includegraphics[width=\textwidth]{Pics/BasisFunc/IntervalFuncs.png}
     \caption{Lagrage order 2 basis functions  on interval $[0,1]$}
     \label{InterFuncs}
\end{figure}

\subsection{Triangle (2D)}
For an order $2$ Lagrange element on a triangle we have $i \in \{0, 1, 2,3,4,5\}$ and the basis functions are of the form
\begin{equation}
\phi_i(x,y) = \xi_{i0} + \xi_{i1} x + \xi_{i2} y + \xi_{i3} xy + \xi_{i4} x^2 + \xi_{i5}y^2,
\end{equation}
where the domain is $x+y\leq 1, x \geq 0$ and $y\geq 0$. For $\ell_i$ we get
\begin{align}
\ell_0 &= (0,0), &\ell_3 = (1/2,1/2), \\
\ell_1 &= (1,0), &\ell_4 = (0,1/2), \\
\ell_2 &= (0,1), &\ell_5 = (1/2,0),
\end{align}
This leads to the set of equations
\begin{align}
\phi_i(\ell_0) &= \xi_{i0}, \\
\phi_i(\ell_1) &= \xi_{i0} + \xi_{i1} + \xi_{i4}, \\
\phi_i(\ell_2) &= \xi_{i0} + \xi_{i2} + \xi_{i5}, \\
\phi_i(\ell_3) &= \xi_{i0} + \frac{\xi_{i1}+\xi_{i2}}{2} + \frac{\xi_{i3} + \xi_{i4} + \xi_{i5}}{4}, \\
\phi_i(\ell_4) &= \xi_{i0} + \frac{\xi_{i2}}{2} + \frac{\xi_{i5}}{4}, \\
\phi_i(\ell_5) &= \xi_{i0} + \frac{\xi_{i1}}{2} + \frac{\xi_{i4}}{4},
\end{align}
Therefore, using the equations created by (\ref{CG_eq}) we get the matrix of coficeits
\begin{align}
\left[ \begin{matrix}
\xi_{00} & \xi_{01} & \xi_{02} & \xi_{03} & \xi_{04} & \xi_{05}\\
\xi_{10} & \xi_{11} & \xi_{12} & \xi_{13} & \xi_{14} & \xi_{25}\\
\xi_{20} & \xi_{21} & \xi_{22} & \xi_{23} & \xi_{24} & \xi_{25}\\
\xi_{30} & \xi_{31} & \xi_{32} & \xi_{33} & \xi_{34} & \xi_{35}\\
\xi_{40} & \xi_{41} & \xi_{42} & \xi_{43} & \xi_{44} & \xi_{45}\\
\xi_{50} & \xi_{51} & \xi_{52} & \xi_{53} & \xi_{54} & \xi_{55}
\end{matrix} \right] &= 
\left[ \begin{matrix}
1 & 1 & 1 & 1 & 1 & 1\\
0 & 1 & 0 & 1/2 & 0 & 1/2\\
0 & 0 & 1 & 1/2 & 1/2 & 0\\
0 & 0 & 0 & 1/4 & 0 & 0\\
0 & 1 & 0 & 1/4 & 0 & 1/4\\
0 & 0 & 1 & 1/4 & 1/4 & 0
\end{matrix} \right]^{-1}, \\
\left[ \begin{matrix}
\xi_{00} & \xi_{01} & \xi_{02} & \xi_{03} & \xi_{04} & \xi_{05}\\
\xi_{10} & \xi_{11} & \xi_{12} & \xi_{13} & \xi_{14} & \xi_{25}\\
\xi_{20} & \xi_{21} & \xi_{22} & \xi_{23} & \xi_{24} & \xi_{25}\\
\xi_{30} & \xi_{31} & \xi_{32} & \xi_{33} & \xi_{34} & \xi_{35}\\
\xi_{40} & \xi_{41} & \xi_{42} & \xi_{43} & \xi_{44} & \xi_{45}\\
\xi_{50} & \xi_{51} & \xi_{52} & \xi_{53} & \xi_{54} & \xi_{55}
\end{matrix} \right] &= 
\left[ \begin{matrix}
1 & -3 & -3 & 4 & 2 & 2\\
0 & -1 & 0 & 0 & 2 & 0\\
0 & 0 & -1 & 0 & 0 & 2\\
0 & 0 & 0 & 4 & 0 & 0\\
0 & 0 & 4 & -4 & 0 & -4\\
0 & 4 & 0 & -4 & -4 & 0
\end{matrix} \right]
\end{align}
Thus leading to basis functions
\begin{align}
\phi_0(x,y) &= 2(x+y)^2 - 3(x+y) + 1, \\
\phi_1(x,y) &= x(2x-1), \\
\phi_2(x,y) &= y(2y-1), \\
\phi_3(x,y) &= 4xy, \\
\phi_4(x,y) &= 4y(1-x-y), \\
\phi_5(x,y) &= 4x(1-x-y),
\end{align}
With the visual representation in Figure \ref{triBasisFuncs}.
\begin{figure}[H]
 \begin{subfigure}{0.5\textwidth}
     \includegraphics[width=\textwidth]{Pics/BasisFunc/triBasis0.png}
     \caption{\phi_0(x,y) = 2(x+y)^2 - 3(x+y) + 1,}
 \end{subfigure}
 \hfill
 \begin{subfigure}{0.5\textwidth}
     \includegraphics[width=\textwidth]{Pics/BasisFunc/triBasis1.png}
     \caption{\phi_1(x,y) = x(2x-1),}
 \end{subfigure}
 \hfill
 \begin{subfigure}{0.5\textwidth}
     \includegraphics[width=\textwidth]{Pics/BasisFunc/triBasis2.png}
     \caption{\phi_2(x,y) = y(2y-1),}
 \end{subfigure}
 \hfill
 \begin{subfigure}{0.5\textwidth}
     \includegraphics[width=\textwidth]{Pics/BasisFunc/triBasis3.png}
     \caption{\phi_3(x,y) = 4xy,}
 \end{subfigure}
 \hfill
 \begin{subfigure}{0.5\textwidth}
     \includegraphics[width=\textwidth]{Pics/BasisFunc/triBasis4.png}
     \caption{\phi_4(x,y) = 4y(1-x-y),}
 \end{subfigure}
 \hfill
 \begin{subfigure}{0.5\textwidth}
     \includegraphics[width=\textwidth]{Pics/BasisFunc/triBasis5.png}
     \caption{\phi_5(x,y) = 4x(1-x-y),}
 \end{subfigure}
 \hfill
 \caption{Basis Functions for Order 2 Lagrange Finite Element on a Triangle.} \label{triBasisFuncs}
\end{figure}

\section{Example 1}
We will solve the PDE (\ref{PDE}) with $\Omega = [0,1]^2, \Gamma_D = \{y=0 \text{ or } y=1\}$ and $\Gamma_N = \{x=0 \text{ or } x=1\}$. We chose a magnetic field such
\begin{equation} \label{E1_b}
\mathbf{b} = \frac{\mathbf{B}}{|\mathbf{B}|}, 
\mathbf{B} = \left[ \begin{matrix}
\alpha(2y-1)cos(m\pi x) + \pi\\
\pi \alpha m (y^2-y)sin(m \pi x)
\end{matrix} \right]
\end{equation}
Thus we have $\Gamma_{in} := \{x=0\}$ and exact solution
\begin{equation} \label{E1_u}
u = \sin(\pi y + \alpha (y^2-y) \cos(m\pi x)) + \varepsilon \cos(2 \pi x) \sin(\pi y)
\end{equation}
This test problem came from \cite{DN}. However, the same or similar problem is very popular and appears in papers \cite{LINE_INT}, \cite{AP}, \cite{MMAP} and \cite{STAB}. We will cover three sub-examples with $(\alpha=0),(\alpha=2,m=1)$ and $(\alpha = 2, m=10)$. Thus we get electromagnetic fields  shown in Figure \ref{E1_VFs}. The field for $\alpha = 0$ is not shown as it is a simple vector field with $\mathbf{b}= [1, 0]$.
\begin{figure}[H]
 \begin{subfigure}{0.5\textwidth}
     \includegraphics[width=\textwidth]{Pics/VectorField/E1b_a2_m1.png}
     \caption{$\alpha=2, m=1$}
 \end{subfigure}
   \begin{subfigure}{0.5\textwidth}
     \includegraphics[width=\textwidth]{Pics/VectorField/E1b_a2_m10.png}
     \caption{$\alpha=2, m=10$}
 \end{subfigure}
 \caption{Electromagetic vector field $\mathbf{b}$ from (\ref{E1_b})} \label{E1_VFs}
\end{figure}
In Figure \ref{E1_us} we have the exact solution $u$ (\ref{E1_u}) to example $1$. We have the source term $f$ is too complicated to write analytically so the visualisation is given in Figure \ref{E1_fs}. We analysis the equation with $\varepsilon = 0.1$ and $\varepsilon = 10^{-10}$. We notice the $\varepsilon = 0$ and $\varepsilon = 10^{-10}$ variants are visually similar for the source term $f$ and exact term $u$ this suggests to consider a method discussed in section \ref{MAP_f}. The main difference is the solution $u$ is not constant on streamlines of the vector field $\mathbf{b}$ this makes logical sense because increasing the anisotripc strength (make $\varepsilon$ smaller) leads to less diffusion perpendicular to the magnetic field and undisturbed diffusion parralle to the magnetic field.
\begin{figure}[H]
 \begin{subfigure}{0.44\textwidth}
     \includegraphics[width=\textwidth]{Pics/uf/U_E1a_eps1.png}
     \caption{$\alpha=0$ and $\varepsilon = 0.1$}
 \end{subfigure}
   \begin{subfigure}{0.44\textwidth}
     \includegraphics[width=\textwidth]{Pics/uf/U_E1a_eps10.png}
     \caption{$\alpha=0$ and $\varepsilon = 10^{-10}$}
 \end{subfigure}
 \begin{subfigure}{0.44\textwidth}
     \includegraphics[width=\textwidth]{Pics/uf/U_E1b_eps1.png}
     \caption{$\alpha=2, m=1$ and $\varepsilon = 0.1$}
 \end{subfigure}
 \begin{subfigure}{0.44\textwidth}
     \includegraphics[width=\textwidth]{Pics/uf/U_E1b_eps_10.png}
     \caption{$\alpha=2, m=1$ and $\varepsilon = 10^{-10}$}
 \end{subfigure}
 \begin{subfigure}{0.44\textwidth}
     \includegraphics[width=\textwidth]{Pics/uf/U_E1c_eps_1.png}
     \caption{$\alpha=2, m=10$ and $\varepsilon = 0.1$}
 \end{subfigure}
 \hfill
 \begin{subfigure}{0.44\textwidth}
     \includegraphics[width=\textwidth]{Pics/uf/u_E1c_eps_10.png}
     \caption{$\alpha=2, m=10$ and $\varepsilon = 10^{-10}$}
 \end{subfigure}
 \caption{Exact solution $u$ (\ref{E1_u}) for Example $1$.} \label{E1_us}
\end{figure}

\begin{figure}[H]
 \begin{subfigure}{0.44\textwidth}
     \includegraphics[width=\textwidth]{Pics/uf/F_E1a_eps1.png}
     \caption{$\alpha=0$ and $\varepsilon = 0.1$}
 \end{subfigure}
   \begin{subfigure}{0.44\textwidth}
     \includegraphics[width=\textwidth]{Pics/uf/F_E1a_eps10.png}
     \caption{$\alpha=0$ and $\varepsilon = 10^{-10}$}
 \end{subfigure}
 \begin{subfigure}{0.44\textwidth}
     \includegraphics[width=\textwidth]{Pics/uf/F_E1b_eps1.png}
     \caption{$\alpha=2, m=1$ and $\varepsilon = 0.1$}
 \end{subfigure}
 \begin{subfigure}{0.44\textwidth}
     \includegraphics[width=\textwidth]{Pics/uf/F_E1b_eps_10.png}
     \caption{$\alpha=2, m=1$ and $\varepsilon = 10^{-10}$}
 \end{subfigure}
 \begin{subfigure}{0.44\textwidth}
     \includegraphics[width=\textwidth]{Pics/uf/F_E1c_eps_1.png}
     \caption{$\alpha=2, m=10$ and $\varepsilon = 0.1$}
 \end{subfigure}
 \hfill
 \begin{subfigure}{0.44\textwidth}
     \includegraphics[width=\textwidth]{Pics/uf/f_E1c_eps_10.png}
     \caption{$\alpha=2, m=10$ and $\varepsilon = 10^{-10}$}
 \end{subfigure}
 \caption{Source term $f$ for Example $1$.} \label{E1_fs}
\end{figure}

\begin{figure}[H]
 \begin{subfigure}{0.44\textwidth}
     \includegraphics[width=\textwidth]{Pics/LHSims/E1a_MMAP_AP_PFL2.png}
     \caption{L2 Error $\alpha=0$}
 \end{subfigure}
   \begin{subfigure}{0.44\textwidth}
     \includegraphics[width=\textwidth]{Pics/LHSims/E1a_MMAP_AP_PFH1.png}
     \caption{H1 Error $\alpha=0$}
 \end{subfigure}
 \begin{subfigure}{0.44\textwidth}
     \includegraphics[width=\textwidth]{Pics/LHSims/E1b_MMAP_AP_PFL2.png}
     \caption{L2 Error $\alpha=2, m=1$}
 \end{subfigure}
 \begin{subfigure}{0.44\textwidth}
     \includegraphics[width=\textwidth]{Pics/LHSims/E1b_MMAP_AP_PFH1.png}
     \caption{H1 Error, $\alpha=2, m=1$}
 \end{subfigure}
 \begin{subfigure}{0.44\textwidth}
     \includegraphics[width=\textwidth]{Pics/LHSims/E1c_MMAP_AP_PFL2.png}
     \caption{L2 Error, $\alpha=2, m=10$}
 \end{subfigure}
 \hfill
 \begin{subfigure}{0.44\textwidth}
     \includegraphics[width=\textwidth]{Pics/LHSims/E1c_MMAP_AP_PFH1.png}
     \caption{H1 Error, $\alpha=2, m=10$}
 \end{subfigure}
 \caption{Numerical demonstration for solver $MMAP, AP$ and $PF$ on a $20\times 20$ quadrilateral grid using CG order $2$ for Example $1$, with $x=\varepsilon$ and $y=$ Error.} \label{E1_LH_PF}
\end{figure}

\begin{figure}[H]
 \begin{subfigure}{0.44\textwidth}
     \includegraphics[width=\textwidth]{Pics/LHSims/E1a_MMAP_LM_SPL2.png}
     \caption{L2 Error $\alpha=0$}
 \end{subfigure}
   \begin{subfigure}{0.44\textwidth}
     \includegraphics[width=\textwidth]{Pics/LHSims/E1a_MMAP_LM_SPH1.png}
     \caption{H1 Error $\alpha=0$}
 \end{subfigure}
 \begin{subfigure}{0.44\textwidth}
     \includegraphics[width=\textwidth]{Pics/LHSims/E1b_MMAP_LM_SPL2.png}
     \caption{L2 Error $\alpha=2, m=1$}
 \end{subfigure}
 \begin{subfigure}{0.44\textwidth}
     \includegraphics[width=\textwidth]{Pics/LHSims/E1b_MMAP_LM_SPH1.png}
     \caption{H1 Error, $\alpha=2, m=1$}
 \end{subfigure}
 \begin{subfigure}{0.44\textwidth}
     \includegraphics[width=\textwidth]{Pics/LHSims/E1c_MMAP_LM_SPL2.png}
     \caption{L2 Error, $\alpha=2, m=10$}
 \end{subfigure}
 \hfill
 \begin{subfigure}{0.44\textwidth}
     \includegraphics[width=\textwidth]{Pics/LHSims/E1c_MMAP_LM_SPH1.png}
     \caption{H1 Error, $\alpha=2, m=10$}
 \end{subfigure}
 \caption{Numerical demonstration for solver $MMAP, LM$ and $SP$ on a $20\times 20$ quadrilateral grid using CG order $2$ for Example $1$, with $x=\varepsilon$ and $y=$ Error.} \label{E1_LM_SP}
\end{figure}


Now we numerically simulate the solution with a $20 \times 20$ grid, where each finite element is a square with width $0.05$. The finite element 

The $MMAP$ line is located under the $AP$ line this strongly suggests these methods are equivalent therefore in further simulations we will only consider the $MMAP$ method as the $AP$ method requires significatly more computionaly power to use. Additionally, the $PF$ method performs worse


\begin{table}[]
\resizebox{\textwidth}{!}{
\begin{tabular}{||cccccccc||}
\cline{3-8}
\multicolumn{2}{c|}{L2 Error} & \multicolumn{2}{c|}{$\alpha = 0$} & \multicolumn{2}{c|}{$\alpha = 2, m=1$} & \multicolumn{2}{c|}{$\alpha = 2, m=10$} \\
\hline 
Size          & dof          & PF        & MMAP        & PF        & MMAP        & PF        & MMAP        \\
\hline \hline
$10\times 10$  &  882    &  $1.26\times10^{-4}$ & $1.26\times10^{-4}$  & $1.64\times10^{-1}$ & $2.25\times10^{-4}$ & $4.29\times10^{-2}$ & $3.16\times10^{-1}$\\
\hline
$20\times 20$  &  3362     & $1.58\times 10^{-5}$ & $1.58\times 10^{-5}$ & $4.01\times10^{-2}$ & $2.80\times10^{-5}$ & $2.94\times10^{-2}$ & $1.25\times10^{-1}$\\
\hline
$40\times 40$ &   13122    & $1.97\times 10^{-6}$ & $1.97\times 10^{-6}$ & $1.42\times10^{-3}$ & $3.44\times10^{-6}$ & $4.23\times 10^{-3}$ & $1.29\times10^{-2}$\\
\hline
$80\times 80$  &  51842     & $2.46\times 10^{-7}$ & $2.46\times 10^{-7}$ & $2.38\times10^{-5}$ & $4.25 \times 10^{-7}$ & $9.71\times10^{-4}$ & $9.70\times10^{-4}$\\
\hline
$160\times 160$ & 206082   & $3.09\times 10^{-8}$ & $3.09\times 10^{-8}$ & $1.47\times10^{-6}$ & $5.25 \times 10^{-8}$ & $2.19\times10^{-4}$ & $6.36\times10^{-5}$ \\
\hline
\end{tabular}}
\end{table}


\begin{table}[]
\resizebox{\textwidth}{!}{
\begin{tabular}{||cccccccc||}
\cline{3-8}
\multicolumn{2}{c|}{H1 Error} & \multicolumn{2}{c|}{$\alpha = 0$} & \multicolumn{2}{c|}{$\alpha = 2, m=1$} & \multicolumn{2}{c|}{$\alpha = 2, m=10$} \\
\hline 
Size          & dof          & PF        & MMAP        & PF        & MMAP        & PF        & MMAP        \\
\hline \hline
$10\times 10$  &  882    &  $8.16\times10^{-3}$ & $8.16\times10^{-3}$  & $1.00\times10^{0}$ & $1.42\times10^{-2}$ & $2.57\times10^{1}$ & $4.22\times10^{0}$\\
\hline
$20\times 20$  &  3362     & $2.04\times 10^{-3}$ & $2.04\times 10^{-3}$ & $3.99\times10^{-1}$ & $3.57\times10^{-3}$ & $1.21\times10^{1}$ & $1.99\times10^{0}$\\
\hline
$40\times 40$ &   13122    & $5.11\times 10^{-4}$ & $5.11\times 10^{-4}$ & $4.37\times10^{-2}$ & $8.89\times10^{-4}$ & $4.38\times 10^{-1}$ & $3.20\times10^{-1}$\\
\hline
$80\times 80$  &  51842     & $1.28\times 10^{-4}$ & $1.28\times 10^{-4}$ & $7.23\times10^{-3}$ & $2.21\times 10^{-4}$ & $2.01\times10^{-1}$ & $5.00\times10^{-2}$\\
\hline
$160\times 160$ & 206082   & $3.19\times 10^{-5}$ & $3.19\times 10^{-5}$ & $1.77\times10^{-3}$ & $5.49\times 10^{-5}$ & $9.79\times10^{-2}$ & $ 1.10\times10^{-2}$ \\
\hline
\end{tabular}}
\end{table}

\section{Example 2, Annulus}
We now propose an example on an annulus with closed field lines.
We will solve the PDE (\ref{}) with $\Omega := \{x,y \in \mathbb{R}, 1<x^2+y^2<4\}$, $\Gamma_D := \{x,y \in \mathbb{R}, (x^2+y^2=1 \text{ or } x^2+y^2=4)\}$ and $\Gamma_N := \emptyset$. We chose the magnetic field 
\begin{equation}
\mathbf{b} = \frac{\mathbf{B}}{|\mathbf{B}|} = 
\left[ \begin{matrix}
-y\\
 x
\end{matrix} \right]/\sqrt{x^2+y^2}, 
\mathbf{B} = \left[ \begin{matrix}
-y\\
 x
\end{matrix} \right]
\end{equation}
which is visualised in Figure \ref{E2_VF}
\begin{figure}[H]
 \includegraphics[width=\textwidth]{Pics/VectorField/E2b.png}
  \caption{Electromagnetic field $\mathbf{b}$ for Example $2$.}
 \label{E2_VF}
\end{figure}

From Fig \ref{E2_VF} it can be seen the magnetic field has closed lines. Therefore, for methods which involve $\Gamma_{in}$ we use the stabilisation variants. However, from the numerical investigation of Example 1 we found that we can change $\Gamma_{in}$ to $\Gamma_{out}$ and get similar error. This strongly suggests that $\Gamma_{in}$ can represent a point on each streamline. Thus we will numerically investigate the consequence of setting $\Gamma_{in}:=\{x,y \in \mathbb{R}, x>0, y=0\}$.

We will use solution
\begin{equation}
u = (1+\varepsilon)(x^2 + y^2 -1)(4-x^2-y^2)
\end{equation}
which is visualised with its source term $f$ in Figure \ref{E2_uf} at $\varepsilon = 0.1$.
\begin{figure}[H]
 \begin{subfigure}{0.5\textwidth}
     \includegraphics[width=\textwidth]{Pics/uf/U_E2_eps_1.png}
     \caption{Exact solution $u$ with mesh.}
 \end{subfigure}
   \begin{subfigure}{0.5\textwidth}
     \includegraphics[width=\textwidth]{Pics/uf/F_E2_eps_1.png}
     \caption{Source term $f$.}
 \end{subfigure}
 \caption{Example $2$ with $\varepsilon = 0.1$} \label{E2_uf}
\end{figure}


\begin{figure}[H]
 \begin{subfigure}{0.44\textwidth}
     \includegraphics[width=\textwidth]{Pics/LHSims/E2/E2_NormalL2.png}
     \caption{L2 Error}
 \end{subfigure}
   \begin{subfigure}{0.44\textwidth}
     \includegraphics[width=\textwidth]{Pics/LHSims/E2/E2_NormalH1.png}
     \caption{H1 Error}
 \end{subfigure}
 \begin{subfigure}{0.44\textwidth}
     \includegraphics[width=\textwidth]{Pics/LHSims/E2/E2_INL2.png}
     \caption{L2 Error, with constraint}
 \end{subfigure}
 \begin{subfigure}{0.44\textwidth}
     \includegraphics[width=\textwidth]{Pics/LHSims/E2/E2_INH1.png}
     \caption{H1 Error, with constraint}
 \end{subfigure}
 \begin{subfigure}{0.44\textwidth}
     \includegraphics[width=\textwidth]{Pics/LHSims/E2/E2_STABL2.png}
     \caption{L2 Error, with stabilisation}
 \end{subfigure}
 \hfill
 \begin{subfigure}{0.44\textwidth}
     \includegraphics[width=\textwidth]{Pics/LHSims/E2/E2_STABH1.png}
     \caption{H1 Error, with stabilisation}
 \end{subfigure}
 \caption{Numerical demonstration for solver $MMAP, MMAP_STAB, PF$ and $PF_STAB$ on a $20\times 20$ quadrilateral grid using CG order $2$ for Example $2$, with $x=\varepsilon$ and $y=$ Error.} \label{E2_LH}
\end{figure}

Therefore, these results strongly suggests that we can apply the condition  will investigate the numerical solution when $\Gamma_{in}$ is applied to part of the streamline, as in the numerical

Do u and f plot for different eps (4)
Show error plots for example for ranging eps  L and H (6)
Use MMAP, Limit, SP, PF
Redo with stab variants (6)
Show stabalisation gives different MMAP and Stab and streamline
Do DN plot (1)

\section{Example 3, Magnetic Islands}
We now solve an example on a square domain with non-zero boundary conditions on $\Gamma_D$ and with closed and open field lines. We will solve the PDE (\ref{PDE}) with $\Omega = [0,1]^2, \Gamma_D = \{y=0 \text{ or } y=1\}$ and $\Gamma_N = \{x=0 \text{ or } x=1\}$. We choose a magnetic field such 
\begin{equation}
\mathbf{b} = \frac{\mathbf{B}}{|\mathbf{B}|}, 
\mathbf{B} = \left[ \begin{matrix}
-\cos(\pi y)\\
4a \sin(4 \pi x)
\end{matrix} \right]
\end{equation}
which is visualised in Figure \ref{E3_VF}. This example was proposed by \cite{DN} but it does not satisfy the restrictions we impose. The restrictions are $\mathbf{b} \cdot \mathbf{n} = 0$ on $\Gamma_D$ this failed restriction can be seen in Figure \ref{VF_E3} and $u = 0$ on $\Gamma_D$ this failed restriction can be seen in Figure \ref{E3_u}. Thus, our numerical solvers will struggle to solve this example, this problem is rectified by in Example $4$. Additionally, \cite{DN} rectified this problem by defining $\Gamma_D := \{\mathbf{x} \in \partial \Omega :  \mathbf{b} \cdot \mathbf{n} = 0\}$.
\begin{figure}[H]
 \includegraphics[width=\textwidth]{Pics/VectorField/E3b.png}
  \caption{Electromagnetic field $\mathbf{b}$ for Example $3$.}
 \label{E3_VF}
\end{figure}

We will use exact solution
\begin{equation}
u = \sin(10\sin(\pi y)-10a\cos(4 \pi x)) + \varepsilon \cos(2 \pi x)\sin(10 \pi y)
\end{equation}
which is visualised with its source term $f$ in Figure \ref{E3_uf} at $\varepsilon = 10^{-10}$. Additionally, we visualise this for $\varepsilon = 0.1$.
\begin{figure}[H]
 \begin{subfigure}{0.5\textwidth} \label{E3_u}
     \includegraphics[width=\textwidth]{Pics/uf/U_E3_eps_10.png}
     \caption{Exact solution $u$.}
 \end{subfigure}
   \begin{subfigure}{0.5\textwidth}
     \includegraphics[width=\textwidth]{Pics/uf/F_E3_eps_10.png}
     \caption{Source term $f$.}
 \end{subfigure}
 \caption{Example $3$ with $\varepsilon = 10^{-10}$.} \label{E3_uf}
\end{figure}
\begin{figure}[H]
 \begin{subfigure}{0.5\textwidth}
     \includegraphics[width=\textwidth]{Pics/uf/U_E3_ep1.png}
     \caption{Exact solution $u$.}
 \end{subfigure}
   \begin{subfigure}{0.5\textwidth}
     \includegraphics[width=\textwidth]{Pics/uf/F_E3_eps_1.png}
     \caption{Source term $f$.}
 \end{subfigure}
 \caption{Example $3$ with $\varepsilon = 0.1$.} \label{E3_uf_01}
\end{figure}


\begin{figure}[H]
 \begin{subfigure}{0.5\textwidth}
     \includegraphics[width=\textwidth]{Pics/ErrorPlots/E3_MMAP_STAB.png}
     \caption{L2 $=2.6\times10^{-2}$ for $MMAP\_STAB$}
 \end{subfigure}
   \begin{subfigure}{0.5\textwidth}
     \includegraphics[width=\textwidth]{Pics/ErrorPlots/E3_PF_STAB.png}
     \caption{L2 $=2.6\times10^{-2}$ for $PF\_STAB$}
 \end{subfigure}
 \caption{Visualisation of Error $u_e-u_h$ for Example $3$ with $\varepsilon = 10^{-10}$, CG order $2$ on a $80 \times 80$ quadralatoral grid.} \label{E3_Error}
\end{figure}


\begin{figure}[H]
 \begin{subfigure}{0.5\textwidth}
     \includegraphics[width=\textwidth]{Pics/ErrorPlots/E3_MMAP_STAB_Q.png}
     \caption{Solves $q$ defined in equation (\ref{MMAP_STAB_w}) for $MMAP\_STAB$}
 \end{subfigure}
   \begin{subfigure}{0.5\textwidth}
     \includegraphics[width=\textwidth]{Pics/ErrorPlots/E3_PF_STAB_Q.png}
     \caption{Solves $q$ defined in equation (\ref{PF_STAB_w}) for $PF\_STAB$}
 \end{subfigure}
 \caption{Visualisation of $q$ defined in sections \ref{MMAP_STAB} and \ref{PF_STAB} for Example $3$ with $\varepsilon = 10^{-10}$, CG order $2$ on a $80 \times 80$ quadrilateral grid.} \label{E3_Q}
\end{figure}



Do u and f plot for different eps (4)
Use MMAP, Limit, SP, PF
with stab variants (6)
Do DN plot (1)
Changing grid size table 
size | dof | L2 error | H1 error PF, MMAP

\section{Example 4, Magnetic Islands}

We now introduce an example on a unit square where $\Gamma_D$ has zero boundary conditions. This is similar to example $3$ but example $3$ failed to satisfy $\mathbf{b}\cdot\mathbf{n} = 0$ on $\Gamma_D$ this can be visually seen from Figure \ref{E3_VF} showing its vector field as $\mathbf{b}$ on $y=0$ and $y=1$ is not parallel to the boundary. The vector field for this example shown in Figure \ref{E4_VF} is parallel to the $\Gamma_D$ boundary. Additionally, this has open and closed field lines.
Thus on $\Omega = [0,1]^2, \Gamma_D = \{y=0 \text{ or } y=1\}$ and $\Gamma_N = \{x=0 \text{ or } x=1\}$ we solve the PDE (\ref{PDE}) with the same magnetic
\begin{equation}
\mathbf{b} = \frac{\mathbf{B}}{|\mathbf{B}|}, 
\mathbf{B} = \left[ \begin{matrix}
-\cos(\pi y)\\
4a \sin(4 \pi x) \sin(\pi y)
\end{matrix} \right]
\end{equation}
where the electromagnetic field $\mathbf{b}$ is shown in Figure \ref{E4_VF}.
\begin{figure}[H]
 \includegraphics[width=\textwidth]{Pics/VectorField/E4b.png}
  \caption{Electromagnetic field $\mathbf{b}$ for Example $4$.}
 \label{E4_VF}
\end{figure}

Where the exact solution is 
\begin{equation}
u = \sin(10 \sin(\pi y) \exp^{-a\cos(4 \pi x)}) 
\end{equation}
which is shown in Figure \ref{E4_uf}

\begin{figure}[H]
 \begin{subfigure}{0.5\textwidth}
     \includegraphics[width=\textwidth]{Pics/uf/U_E4_eps_10.png}
     \caption{Exact solution $u$.}
 \end{subfigure}
   \begin{subfigure}{0.5\textwidth}
     \includegraphics[width=\textwidth]{Pics/uf/F_E4_eps_10.png}
     \caption{Source term $f$.}
 \end{subfigure}
 \caption{Example $4$ with $\varepsilon = 10^{-10}$.} \label{E4_uf}
\end{figure}

\begin{figure}[H]
 \begin{subfigure}{0.5\textwidth}
     \includegraphics[width=\textwidth]{Pics/LHSims/E4/E4_STABL2.png}
     \caption{L2 Error}
 \end{subfigure}
   \begin{subfigure}{0.5\textwidth}
     \includegraphics[width=\textwidth]{Pics/LHSims/E4/E4_STABH1.png}
     \caption{H1 Error}
 \end{subfigure}
 \caption{Numerical solution of Example $4$ for solver $MMAP\_STAB$ and $PF\_STAB$ with varying $\varepsilon$ on $20 \times 20$ quadrilateral grid with CG order $2$. This has $3362$ degrees of freedom.} \label{E4_eps}
\end{figure}

\begin{table}[H]
\resizebox{\textwidth}{!}{
\begin{tabular}{||cccccc||}
\cline{3-6}
\multicolumn{2}{c|}{Error} & \multicolumn{2}{c|}{L2 Error} & \multicolumn{2}{c|}{H1 Error} \\
\hline 
Size          & dof          & PF        & MMAP        & PF        & MMAP        \\
\hline \hline
$10\times 10$  &  882    &  $2.33\times10^{-2}$ & $3.91\times10^{-2}$  & $1.97\times10^{0}$ & $2.16\times10^{0}$\\
\hline
$20\times 20$  &  3362     & $8.78\times 10^{-3}$ & $1.00\times 10^{-2}$ & $1.16\times10^{0}$ & $1.25\times10^{0}$\\
\hline
$40\times 40$ &   13122    & $1.14\times 10^{-3}$ & $1.26\times 10^{-3}$ & $2.98\times10^{-1}$ & $3.31\times10^{-1}$\\
\hline
$80\times 80$  &  51842     & $1.45\times 10^{-4}$ & $1.56\times 10^{-4}$ & $7.57\times10^{-2}$ & $8.32\times 10^{-2}$\\
\hline
$160\times 160$ & 206082   & $1.84\times 10^{-5}$ & $1.92\times 10^{-5}$ & $1.93\times10^{-2}$ & $2.04\times 10^{-2}$ \\
\hline
\end{tabular}}
\end{table}

\begin{figure}[H]
 \begin{subfigure}{0.5\textwidth}
     \includegraphics[width=\textwidth]{Pics/ErrorPlots/E4_MMAP_STAB.png}
     \caption{L2 $=1.6\times10^{-4}$ for $MMAP\_STAB$}
 \end{subfigure}
   \begin{subfigure}{0.5\textwidth}
     \includegraphics[width=\textwidth]{Pics/ErrorPlots/E4_PF_STAB.png}
     \caption{L2 $=1.4\times10^{-4}$ for $PF\_STAB$}
 \end{subfigure}
 \caption{Visualisation of Error $u_e-u_h$ for Example $4$ with $\varepsilon = 10^{-10}$, CG order $2$ on a $80 \times 80$ quadrilateral grid.} \label{E4_Error}
\end{figure}

\begin{figure}[H]
 \begin{subfigure}{0.5\textwidth}
     \includegraphics[width=\textwidth]{Pics/ErrorPlots/E4_MMAP_STAB_Q.png}
     \caption{Solves $q$ defined in equation (\ref{MMAP_STAB_w}) for $MMAP\_STAB$}
 \end{subfigure}
   \begin{subfigure}{0.5\textwidth}
     \includegraphics[width=\textwidth]{Pics/ErrorPlots/E4_PF_STAB_Q.png}
     \caption{Solves $q$ defined in equation (\ref{PF_STAB_w}) for $PF\_STAB$}
 \end{subfigure}
 \caption{Visualisation of $q$ defined in sections \ref{MMAP_STAB} and \ref{PF_STAB} for Example $4$ with $\varepsilon = 10^{-10}$, CG order $2$ on a $80 \times 80$ quadrilateral grid.} \label{E3_Q}
\end{figure}

\section{Formulation of Toroidal Coordinates}
Here we discuss how to parameterise a torus and to find its inverted map. We use the Figures \ref{Torus_XY} and \ref{Torus_z} below to help us derive the parameterisation.
\begin{figure}[H]
 \begin{subfigure}{0.5\textwidth}
     \includegraphics[width=\textwidth]{Pics/TorusCordsXY.png}
     \caption{Cross section of torus with $z=0$}
     \label{Torus_XY}
 \end{subfigure}
 \hfill
 \begin{subfigure}{0.5\textwidth}
     \includegraphics[width=\textwidth]{Pics/TorusCordsr_XYZ.png}
     \caption{Cross section of torus}
     \label{Torus_z}
 \end{subfigure}
 \caption{Visual aids for the derivation of the Toroidal parameterisation.} \label{BO}
\end{figure}

From visual inspection of Figure \ref{Torus_XY} we get 
\begin{align}
x &= (r_I + r_{xy})\cos(\theta)\\
y &= (r_I + r_{xy})\sin(\theta)
\end{align}
Additionally, from Figure \ref{Torus_z} we get 
\begin{align}
r_{xy}& = r \cos(\phi) \\
z &= r \sin(\phi)
\end{align}
Thus joining the equations (\ref{}) to (\ref{}) we get the toroidal parametrisation
\begin{align}
x &= (r_I + r\cos(\phi))\cos(\theta) \\
y &= (r_I + r \cos(\phi))\sin(\theta) \\
z &= r \sin(\phi)
\end{align}
where $0\leq r \leq r_O$ and $0 \leq \phi, \theta \leq 2 \pi$. Now we calculate the inverse of this map by considering $x^2 + y^2 + z^2$. 
\begin{align}
x^2 + y^2 + z^2 &= (r_I + r \cos(\phi))^2 + r^2\sin^2(\phi)\\
&= r_I^2 + 2r_Ir\cos(\phi) + r^2\\
&= r_I^2 +2r_I \sqrt{r^2-z^2} + r^2
\end{align}
This is a quadratic in disguise thus after some algebraic manipulation we get $4$ possible solutions
\begin{equation}
r = \pm\sqrt{r_I^2 + x^2 + y^2 + z^2 \pm 2r_I\sqrt{x^2+y^2}}
\end{equation}
However, by using enforcing $r\geq0$ we remove two solutions. We find the final solution by substitution. We use the substitution $(x, y, z) = (r_I, 0, 0)$ where $r=0$. With $+$ we get $r=2r_I$ and for $-$ we get $r = 0$. Thus we have 
\begin{equation}
r = \sqrt{r_I^2 + x^2 + y^2 + z^2 - 2r_I\sqrt{x^2+y^2}}
\end{equation}
For completeness will we calculate $\theta$ and $\phi$. To get $\theta$ and $\phi$ we do a similar process used for calculating the argument of complex numbers. We note $r_{xy}=x^2+y^2-r_I$ thus we get
\begin{align}
\theta &= \text{atan2}(y, x) \\
\phi &= \text{atan2}(z, x^2+y^2-r_{I})
\end{align}
where atan2 is a common variation of the arctan function.

We use the inverse of the map because calculating the differential operators leads to large expressions. These expressions can be found in \cite{}. When we need to solve a PDE on a domain which can be parameterised it is usually easier to turn the source term $f$ into Cartesian coordinates so we deal with the Cartesian differential operators.

\section{Example 5, Torus}
We now look at an example in 3D where the domain $\Omega $ is a torus with inner radius $r_I = 1$, outer radius $r_O = 0.5$ and is centred at the origin. It will have zero Dirichlet boundary conditions. For this eaxmple we will use Toroidal coordinates which are explained in Appendix \ref{} and demonstrate how to find the inverse map. We invert back to cartesian cordinates because the toroidal laplacian \cite{} is very complicated. Thus the definition of $r$ is
\begin{equation}
r = \sqrt{r_I^2 + x^2 + y^2 + z^2 -2r_I\sqrt{x^2+y^2}}
\end{equation}
Where $r$ denotes the minimum distance from the circulfurance of a circle centered at the origin with radius $r_I$ and has $z=0$. We have magnetic field 
\begin{equation}
\mathbf{b} = \frac{\mathbf{B}}{|\mathbf{B}|} = 
\left[ \begin{matrix}
-y\\
 x \\
 0
\end{matrix} \right]/\sqrt{x^2+y^2}, 
\mathbf{B} = \left[ \begin{matrix}
-y\\
 x\\
 0
\end{matrix} \right]
\end{equation}
Also, we will have exact solution
\begin{equation}
u = 
\begin{cases}
(1+\varepsilon)((0.5)^2 - r ),\\
(1+\varepsilon)(0.25 -  \sqrt{r_I^2 + x^2 + y^2 + z^2 -2r_I\sqrt{x^2+y^2}})
\end{cases}
\end{equation}
This $u$ is displayed in Figure \ref{E5_uf} with its source term $f$ and $\varepsilon=10^{-10}$.
\begin{figure}[H]
 \begin{subfigure}{0.5\textwidth}
     \includegraphics[width=\textwidth]{Pics/uf/U_E5_eps_10.png}
     \caption{Exact solution $u$.}
 \end{subfigure}
   \begin{subfigure}{0.5\textwidth}
     \includegraphics[width=\textwidth]{Pics/uf/F_E5_eps_10.png}
     \caption{Source term $f$.}
 \end{subfigure}
 \caption{Example $5$ with $\varepsilon = 10^{-10}$.} \label{E5_uf}
\end{figure}


\begin{figure}[H]
 \begin{subfigure}{0.5\textwidth}
     \includegraphics[width=\textwidth]{Pics/LHSims/E5/E5_STABL2.png}
     \caption{L2 Error}
 \end{subfigure}
   \begin{subfigure}{0.5\textwidth}
     \includegraphics[width=\textwidth]{Pics/LHSims/E5/E5_STABH1.png}
     \caption{H1 Error}
 \end{subfigure}
 \caption{Numerical solution of Example $5$ for solver $MMAP\_STAB$ and $PF\_STAB$ with varying $\varepsilon$ on torus mesh with CG order $2$. This has $69976$ degrees of freedom.} \label{E5_eps}
\end{figure}

\chapter{Conclusion and Future Work}

\section{Methods For Investigation}

\subsection{Line Integration}
The paper \cite{LINE_INT} states we can use line integration to solve PDE (\ref{PDE}).  The idea of line integration is to use the fact that the solution should be approximately constant along streamlines for $\varepsilon \ll 1$. Therefore, we only have to calculate the PDE on $\Gamma_{in}$. A proof is located 
We demonstrate the idea of this method for Example $1$ when $\alpha=0$. Thus, the magnetic field $\mathbf{b} = [1, 0]$. When substituting this $\mathbf{b}$ into (\ref{PDE}) we get 
\begin{equation} \label{LINE_PDE}
\begin{cases}
\varepsilon^{-1}u_{xx} + u_{yy} = f(x,y), &\in \Omega,\\
u_x = 0, &\text{on } \Gamma_{N},\\
u = 0, &\text{on } \Gamma_{D}.
\end{cases}
\end{equation}
Now we take line integrals along the vector field and take the limit of $\varepsilon \rightarrow 0$ in (\ref{LINE_PDE}) to get
\begin{equation} \label{LINE_ID1}
\begin{cases}
u_{xx} = 0, &\in \Omega,\\
u_{x} = 0, & \text{on }x=0,\\
-\int_0^1 u_{yy} dx = \int_0^1f(x,y)dx, &\forall y \in [0,1],\\
u = 0, &\text{on } y=0 \text{ or } y=1.
\end{cases}
\end{equation}
This problem can be solved. From the first two equations in (\ref{LINE_ID1}) we get the solution for $u$ is constant along $\mathbf{b}$. Thus $u$ is independent of $x$. From this fact and last the two equations we get
\begin{equation}
\begin{cases}
-u_{yy} =\int_0^1 f(x, y) dx, &\text{on } x=0,\\
u = 0, &\text{on } y=0 \text{ or } y=1,
\end{cases}
\end{equation}
Now we demonstrate this works. For Example 1 with $\alpha = 0$ we get 
\begin{equation}
f(x,y) = \sin(\pi y)\pi^2(1+(4+\varepsilon)\cos(2\pi x)) 
\end{equation}
Thus after some integration we get
\begin{equation}
u_{yy} = - \int_0^1 f(x,y)dx = - \int_0^1 \sin(\pi y)\pi^2(1+(4+\varepsilon)\cos(2\pi x)) dx = -\pi^2 sin(\pi y)
\end{equation}
Which implies $u= \sin(\pi y)$, this is missing the $\varepsilon$ order term from (\ref{E1_u}) because we have solved the limit problem. Therefore we get the value of $u$
 as $\varepsilon \rightarrow 0$. Additionally, the paper \cite{LINE_INT} describes how to use this method for a vector field of the form $\mathbf{b} = [\cos(\theta), \sin(\theta)]$. Also, as Example $1$ is a popular example to use to demonstrate methods for solving PDE (\ref{PDE}), they cover $(\alpha = 0)$ and $(\alpha=2, m=1)$ in their numerical demonstrations. Additionally, our $(PF)$ and $(MMAP)$ seem to have smaller L2 Error for the same examples, this is hard to compare because they used a finite difference method to numerically solve their PDE.
 
 However, it is not clear how a stabilisation technique can be used to solve vector fields with closed field lines. At the end of their paper it stated their future would solve this problem. One potential solution could be to take a point on each stream line and then join the points together with respect to which streamlines are touching. Then solve the reduced dimension PDE.

\subsection{Mapping of $f$} \label{MAP_f}
The idea behind this method is given $f$ for PDE (\ref{PDE}) when $\varepsilon = 10^{-15}$. We use this $f$ and solve the PDE for $\varepsilon = 10^{-1}$, this reduces the anisotropic strength and should reduce the error. We use the code below to quickly test this hypothesis on Example $4$ using the $(MMAP\_STAB)$ method.

Table \ref{Tbl_Map_f} contains some output of the code. It shows doing this method slightly reduces H1 Error but slightly increases L2 Error. Therefore, this suggests this technique does not work. Also, by looking at our numerical demonstrations of the $(MMAP\_STAB)$ method shown in Figure \ref{E4_eps} it can be seen varying $\varepsilon$ does not change the error. This shows that the for the $(MMAP\_STAB)$ method the difficulty comes from having a vector field $\mathbf{b}$ that is highly anisotropic. 
\begin{table}[H]
\begin{center}
\begin{tabular}{||ccc||}
\cline{2-3}
\multicolumn{1}{c|}{$\varepsilon$} & \multicolumn{1}{c|}{L2 Error} & \multicolumn{1}{c|}{H1 Error} \\
\hline 
\hline
$10^{-15}$  &  $0.02880$   &  $0.34581$\\
$10^{-10}$  &  $0.02880$   &  $0.34581$\\
$10^{-5}$  &  $0.02880$   &  $0.34564$\\
$10^{-4}$  &  $0.02880$   &  $0.34415$\\
$10^{-3}$  &  $0.02880$   &  $0.31149$\\
$10^{-2}$  &  $0.02880$   &  $0.31481$\\
$10^{-1}$  &  $0.02884$   &  $0.31149$\\
$10^{0}$  &  $0.02904$   &  $0.31333$\\
\hline
\end{tabular}
\end{center}
\caption{Error for solving with $(MMAP\_STAB)$ with $\varepsilon$ on a $40 \times 40$ quadrilateral grid with an order $2$ Lagrange Element (dof$=13122$) using $f$ from $\varepsilon=10^{-15}$.}
\label{Tbl_Map_f}
\end{table}
Future work would involve finding a mapping for the $f$ generated by the PDE (\ref{PDE}) when $\varepsilon=10^{-15}$ and the $f$ generated by the PDE when $\varepsilon=10^{-1}$. This would lead to a reduced error because running the numerical simulation for Example $4$ on a $40 \times 40$ quadrilateral grid with order $2$ Lagrange elements with $\varepsilon = 0.1$ we get an L2 Error of $0.001249$.

\subsection{Splitting of Domain}


\subsection{More Finite Elements}
Mainlt one which cont on derivative. (Hermite)

\subsection{Parallel Implementation}
It is clear that a parallel implementation would speed up the calculation of the solution. Thus in the same amount of time as the sequential Firedrake implementation a higher density mesh can be used and this leads to a solution with less error.

When using a local formulation it is trivial to calculate and add the contribution from each finite element to the global matrix in parallel.

What is not so clear is how to solve a linear system in parallel. 

What takes the most time in FEM

Future work would involve a git branch 

\section{Conclusion}
Check out github page \cite{Hub}

\printbibliography[heading=bibintoc]

\appendix

\section{Title of Appendix}

Appendices are definitely not necessary and assessors are not obliged to read them so only use them for non-vital text, figures or calculations.$\phi$




\end{document}